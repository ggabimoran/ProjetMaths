\documentclass{exam}
\usepackage[utf8]{inputenc}
\usepackage{systeme,mathtools}
\usepackage{amssymb,amsmath,amsthm}
\usepackage[francais]{babel}
\usepackage{fontspec}
\usepackage{systeme}
\begin{document}

8) Le système à la date $t_N$ vaut :
\newline
\vspace{0.5cm}
$\alpha_{N-1}(S_{t_{N-1}}^{(N)})(1 + h_N)S_{t_{N-1}}^{(N)} + \beta_{N-1}(S_{t_{N-1}}^{(N)})S_{t_N}^{0} = f((1+h_N)S_{t_{N-1}}^{(N)})$ \hspace{1cm} [1]
\newline
\vspace{0.5cm}
$\alpha_{N-1}(S_{t_{N-1}}^{(N)})(1+b_N)S_{t_{N-1}}^{(N)} + \beta_{N-1}(S_{t_{N-1}}^{(N)})S_{t_N}^{0} = f((1+b_N)S_{t_{N-1}}^{(N)})$ \hspace{1cm} [2]
\newline
\vspace{0.5cm}
$[1]-[2]$ donne :
\newline
\vspace{0.5cm}
$\alpha_{N-1}(S_{t_{N-1}}^{(N)})=\frac{f((1+h_N)S_{t_{N-1}}^{(N)})-f((1+b_N)S_{t_{N-1}}^{(N)})}{(h_N-b_N)S_{t_{N-1}}^{(N)}}$
\newline
\vspace{0.5cm}
$(1+h_N)[1]-(1+b_N)[2]$ donne:
\newline
\vspace{0.5cm}
$\beta_{N-1}(S_{t_{N-1}}^{(N)})=\frac{1}{S_{t_N}^0(h_N-b_N)}(f((1+b_N)S_{t_{N-1}}^{(N)})(1+h_N)-f((1+h_N)S_{t_{N-1}}^{(N)})(1+b_N))$
\newline
\vspace{0.5cm}
9) Le système à la date $t_k$ vaut :
\newline
\vspace{0.5cm}
$\alpha_{k-1}(S_{t_{k-1}}^{(N)})(1 + h_N)S_{t_{k-1}} + \beta_{k-1}(S_{t_{k-1}}^{(N)})S_{t_k}^{0} = v_k((1+h_N)S_{t_{k-1}}^{(N)})$ \hspace{1cm} [1]
\newline
\vspace{0.5cm}
$\alpha_{k-1}(S_{t_{k-1}}^{(N)})(1+b_N)S_{t_{k-1}} + \beta_{k-1}(S_{t_{k-1}}^{(N)})S_{t_k}^{0} = v_k((1+b_N)S_{t_{k-1}}^{(N)})$ \hspace{1cm} [2]
\newline
\vspace{0.5cm}
Ce qui donne, par un raisonnement analogue : 
\newline
\vspace{0.5cm}
$\alpha_{k-1}(S_{t_{k-1}}^{(N)})=\frac{v_k((1+h_N)S_{t_{k-1}}^{(N)})-v_k((1+b_N)S_{t_{k-1}}^{(N)})}{(h_N-b_N)S_{t_{k-1}}^{(N)}}$
\newline
\vspace{0.5cm}
$\beta_{k-1}(S_{t_{k-1}}^{(N)})=\frac{1}{S_{t_k}^0(h_N-b_N)}(v_k((1+b_N)S_{t_{k-1}}^{(N)}(1+h_N)-v_k((1+h_N)S_{t_{k-1}}^{(N)}(1+b_N))$
\newline
\vspace{0.5cm}
10) Voir python.
\newline
\vspace{0.5cm}


\end{document}
